\section{Introduction}
\subsection{Nonhomogeneous Linear Differential Equations}
This text is concerned with the solutions to non-homogeneous linear differential equations, which have the form
\begin{equation}
    \L u=\phi,
\end{equation}
over an interval \(\lima \leq x \leq \limb\) and subject to certain boundary conditions, where \(\L\) is an \(n\)th order linear ordinary differential operator and where the function \(\phi\) is integrable on the given interval.\footnote{For \(\L\) to be linear, it must satisfy the condition
\begin{equation}\label{eq:linearity}
	\L(\alpha v + \beta w) = \alpha \L v + \beta \L w
\end{equation}
for arbitrary functions \(v\) and \(w\), with \(\alpha\) and \(\beta\) being constant.}
We begin by proving a theorem about such operators.


\begin{theorem}
	\(\L\) is linear if and only if it is of the form
	\begin{equation} 
		\L = a_n(x) \frac{d^n}{dx^n} + a_{n-1}(x) \frac{d^{n-1}}{dx^{n-1}} + \cdots + a_0(x).
	\end{equation}
\end{theorem}

\begin{proof}\(\impliedby\)\\
	We prove, by induction, that \(\D^{k}\) is a linear differential operator for all \(k \in \mathbb{N}\). Note: \(f^{(k)}=\D^{k}f\)\\
	\textbf{Base Step} Let \(k=1\).\\
	\begin{equation*}
		\begin{split}
			\D (\a u(x) + \b v(x)) &= \lim_{h\to 0} \frac{\a u(x+h) + \b v(x+h)-(\a u(x) + \b v(x))}{h}\\
			&=\a \frac{u(x+h) - u(x)}{h} + \b \frac{v(x+h)-v(x)}{h}\\
			&= \a \D u + \b \D v
		\end{split}
	\end{equation*}
	\textbf{Induction Step} Suppose that the statement holds for some \(k\in\mathbb{N}\), \(k>1\). Then,
	\begin{equation*}
		\begin{split}
			\D^{k+1} (\a u + \b v) &= \D ( \D^{k}(\a u + \b v))\\
			&=\D(\a \D^{k} u + \b \D^{k} v)\\
			&=\a \D(\D^{k} u) + \b \D (\D^{k} v)\\
			&= \a \D^{k+1} u + \b \D^{k+1} v
		\end{split}
	\end{equation*}
	It follows that 
	\begin{equation*}
		\sum_{k=0}^{n} a_k(x) D^{k}
	\end{equation*}
	is also linear because \(a_k(x) D^{k}\) is linear and a sum of linear operators is linear.\\
	\(\implies\)\\
	Let
	\begin{equation*}
		\begin{split}
			\L u &= f(x, \underbrace{u, u', u'', \dots, u^{(n)}}_\u)\\
			&=f(x, \u).
		\end{split}
	\end{equation*}
	Then,
	\begin{equation*}
		\L (\a v + \b w) = f(x, \a \textbf{v} + \b \textbf{w})
	\end{equation*}
	and
	\begin{equation*}
		\a \L v + \b \L w = \a f(x,\textbf{v}) + \b f(x,\textbf{w}).
	\end{equation*}
	If \(\L\) is linear then 
	\begin{equation*}
			\L(\a v + \b w) = \a \L v + \b \L w\
	\end{equation*}
	or
	\begin{equation*}
		f(x,\a \textbf{v} + \b \textbf{w})= \a f(x, \textbf{v}) + \b f(x, \textbf{u})
	\end{equation*}
	It follows that 
	\begin{equation*}
		f(x, \u + \epsilon \v) = f(x, \u) + \epsilon f(x, \v)
	\end{equation*}
	or
	\begin{equation*}
		\frac{f(x, \u + \epsilon \v)-f(x,\u)}{\epsilon} = f(x,\v).
	\end{equation*}
	Next, we take the limit as \(\epsilon \to 0\),
	\begin{equation}\label{eq:dirDiv}
		\begin{split}
			\lim_{\epsilon \to 0}\frac{f(x, \u + \epsilon \v )-f(x,\u )}{\epsilon} &= \D_\v f(x,\u)\\
			&= \v \cdot \nabla f(x,\u)\\
			&= f(x,\v),
		\end{split}
	\end{equation}
	where \(\D=\langle \frac{\partial}{\partial u}, \frac{\partial}{\partial u'}, \dots, \frac{\partial^(n)}{\partial u^{(n)}\rangle} \)
	Let 
	\begin{equation*}
		\u = \v = \langle 0, \dots, u^{(i)}, 0, \dots, 0 \rangle
	\end{equation*}
	and
	\begin{equation*}
		f(x,u^{(i)}) = f(x, \langle 0, \dots, u^{(i)}, 0, \dots, 0 \rangle ).
	\end{equation*}
	Then it follows from equation (\ref{eq:dirDiv}) that,
	\begin{equation*}
		u^{(i)}\frac{\partial f(x, u^{(i)})}{\partial u^{(i)}} = f(x,u^{(i)}).
	\end{equation*}
	We solve this equation with separation of variables, 
	\begin{equation*}
		\begin{split}
			\frac{\partial f(x, u^{(i)})}{f(x,u^{(i)})} &= \frac{\partial u^{(i)}}{u^{(i)}}\\
			\int\frac{\partial f(x, u^{(i)})}{f(x,u^{(i)})} &= \int \frac{\partial u^{(i)}}{u^{(i)}}\\
		\end{split}
	\end{equation*}
	Thus, 
	\begin{equation*}
		f(x,u^{(i)}) = a_i(x)u^{(i)}
	\end{equation*} 
	and so 
	\begin{equation*}
		\begin{split}
			f(x,\u) &= f(x, \sum_i \langle 0, 0, \dots, u^{(i)}, \dots, 0, 0 \rangle)\\
			&=\sum_i f(x,u^{(i)}) \\
			&= \sum_i a_i(x)u^{(i)}(x)\\
			&= \left[a_n(x) \frac{d^n}{dx^n} + a_{n-1}(x) \frac{d^{n-1}}{dx^{n-1}} + \cdots + a_0(x)\right]u.
		\end{split}
	\end{equation*}
\end{proof}

	Since \(\L\) is of order \(n\), there will be \(n\) boundary conditions of the general form 
\begin{equation}
	\mathbf{B}_j (u) = c_j;\quad j=1,2,\dots,n,
\end{equation}
where the \(\mathbf{B}_j\)'s are prescribed functionals \footnote{A \df{functional} is a transformation with a set of functions as its domain and a set of numbers as its range. To illustrate, consider the functional 
\begin{equation}
	\mathcal{F}(u) = \int_{0}^{1} u^2(x)dx.
\end{equation}
The domain of this functional might be the set of functions defined over the interval \([0,1]\) and for which the integral of \(u^2\) from 0 to 1 exists. The range is \([0, \infty)\).
} and \(c_j\)'s are prescribed constants. We will only consider \(\mathbf{B}_j\)'s that are linear combinations of \(u\) and its derivatives through order \(n-1\) and evaluated at the endpoints, a and b. 

For \(\mathbf{B}_j\) to be \df{linear}, it must satisfy the condition
\begin{equation}
	\mathbf{B}_j(\alpha v + \beta w) = \alpha \mathbf{B}_j (v) + \beta \mathbf{B}_j(w).
\end{equation}